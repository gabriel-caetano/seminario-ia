

\documentclass[12pt]{article}
\usepackage{sbc-template}
\usepackage{amssymb}
\usepackage{graphicx,url}
\usepackage{multirow}
\usepackage[brazil]{babel}   
\usepackage[utf8]{inputenc}  
\usepackage{lipsum}
\usepackage{float}
\restylefloat{table}

\sloppy

\title{Aprendizado por Transferência na Construção de Modelos Preditivos de Doença Renal Crônica}

\author{ }
\address{}

% \author{Gabriel V. S. Caetano\inst{1}, Isabel Cristina Reinheimer\inst{1,2}, \\ Solana de Melo \inst{2}, Maria Eduarda Silveira Bellinaso\inst{2},\\ Laura Mattiello Ribeiro\inst{2}, Isabela Velho de Quadros\inst{2},\\ Giovana Ajnhorn Mattiello\inst{2}, Rafaela Jung Kurtz Rodrigues\inst{2},\\ Carlos Eduardo Poli de Figueiredo\inst{2},\\ Luis A. L. Silva\inst{1} }


% \address{Departamento de Computação Aplicada, Curso de Ciência da Computação \\ Universidade Federal de Santa Maria (UFSM)\\
%  97.105-970 -- Santa Maria -- RS -- Brasil
% \nextinstitute
%  Escola de Medicina (ESMED), Laboratório de Nefrologia\\
%  Pontifícia Universidade Católica do Rio Grande do Sul (PUCRS)\\ 90.619-900 -- Porto Alegre -- RS -- Brasil
%  \email{\{gvcaetano,luisalvaro\}@inf.ufsm.br, cristinareinheimer@gmail.com}
% }

\begin{document} 

\maketitle

\begin{abstract} 
  This work explores Transfer Learning (TL) to predict Chronic Kidney Disease (CKD) progression using tabular data, an underexplored area in Deep Learning. A Multi-Layer Perceptron (MLP) model was pre-trained on data from 880 elderly patients ($\geq$60 years) and fine-tuned on 258 adult patients ($<$60 years). Compared to baseline models, the TL approach achieved the highest accuracy (88,46\%) and precision (92,31\%), with an F1-Score of 80,0\%. The results are promising and suggest that TL is an effective strategy for developing robust predictive models in clinical settings with limited tabular data.
\end{abstract}

\begin{resumo} 
Este trabalho explora Aprendizado por Transferência (TL) sobre dados tabulares na predição de progressão da Doença Renal Crônica (DRC). Um modelo de redes neurais foi pré-treinado com dados de 880 pacientes idosos ($\geq$60 anos) e ajustado para 258 pacientes adultos ($<$60 anos). Em comparação a um modelo fonte treinado, a abordagem de TL alcançou a maior acurácia (88,46\%) e precisão (92,31\%), com um \textit{F1-Score} de 80,0\%. Os resultados sugerem que o TL é uma estratégia eficaz para desenvolver modelos preditivos robustos em cenários clínicos com dados tabulares limitados.
\end{resumo}


\section{Introdução} \label{sec:introducao}

A Doença Renal Crônica (DRC) constitui um relevante problema de saúde pública, afetando mais de 850 milhões de indivíduos em todo o mundo~\cite{jager2019single}. Caracteriza-se pela perda progressiva (estágios 1 a 5) e irreversível da função renal. O diagnóstico e monitoramento da progressão da DRC dependem fundamentalmente da análise de dados tabulares, incluindo exames laboratoriais, parâmetros clínicos e características demográficas dos pacientes \cite{kdigo:2024}. Neste contexto, a aplicação de técnicas de \textit{Deep Learning - DL} para dados tabulares na área da saúde tem apresentado relevantes resultados para esta área de pesquisa, embora ainda seja um campo relativamente inexplorado quando comparado às aplicações em dados não estruturados, como diagnóstico baseado em imagens.

Estudos recentes evidenciam o potencial do Aprendizado por Transferência (Transfer Learning – TL) \cite{niu2021decade} como uma abordagem promissora para mitigar limitações inerentes ao treinamento e ao fine-tuning de modelos de Inteligência Artificial (IA), especialmente no contexto de conjuntos de dados restritos, característicos de grupos específicos de pacientes e determinadas condições clínicas. Apesar disso, os autores em \cite{hollmann:2025} observam que ainda existe uma lacuna substancial na literatura referente à investigação de TL em dados tabulares na saúde. Este problema assume particular relevância ao se considerar que os dados tabulares constituem a base fundamental para a tomada de decisão clínica em diversas especialidades, incluindo a nefrologia.

O problema analisado neste trabalho reside na dificuldade de desenvolver modelos preditivos robustos para a progressão DRC em populações específicas. Entre os fatores que contribuem para isso, a variabilidade associada à faixa etária representa um desafio adicional para a generalização dos modelos. Do ponto de vista clínico, biológico e fisiopatológico, o envelhecimento está relacionado a alterações estruturais e funcionais dos rins, como redução do número de néfrons, esclerose glomerular e declínio da taxa de filtração glomerular, além de maior carga de comorbidades (como hipertensão e diabetes) que influenciam trajetórias distintas de progressão da DRC entre idosos e adultos. Na prática clínica, a estratificação por idade é particularmente relevante para serviços de saúde, pois permite identificar perfis de risco diferenciados e personalizar recursos diagnósticos e terapêuticos. Diante disso, este trabalho investiga se o conhecimento adquirido por meio do treinamento em uma população idosa com DRC pode ser efetivamente transferido para aprimorar a predição da progressão da doença em uma população adulta, utilizando dados tabulares e técnicas de TL supervisionado em um modelo \textit{Multi-Layer Perceptron} (MLP). 


\section{Metodologia} \label{sec:metodologia}

A metodologia adotada foi estruturada em três etapas: (1) seleção dos atributos relevantes de um \textit{dataset} público \textit{CKD-ROUTE} \cite{iimori:2018} sobre DRC; (2) implementação da estratégia de TL, compreendendo o pré-treino do modelo no \textit{domínio fonte} (pacientes idosos) e subsequente ajuste para o \textit{domínio alvo} (adultos); (3) definição e execução de diferentes cenários experimentais para avaliar a eficácia da proposta.

A amostra, composta por 1.138 indivíduos, apresentou predomínio de homens (69,6\%) e idosos ($\geq$60 anos; 77,3\%). A progressão da DRC (atributo alvo) ocorreu em 24,6\% dos pacientes. Para a avaliação do TL, o \textit{dataset} foi segmentado em um \textit{Subconjunto Fonte} - composto por idosos (n=880), e um \textit{Subconjunto Alvo} - composto de adultos ($\geq$18$<$60 anos; n=258). A distribuição de gênero manteve-se semelhante entre os grupos; entretanto, a taxa de progressão da DRC foi inferior entre os idosos (22,5\%) em comparação aos adultos (31,78\%). É importante ressaltar que a segmentação do \textit{dataset} por idade justifica-se pela plausibilidade clínica, biológica e fisiopatológica que resultam em trajetórias distintas de progressão da DRC, o que sustenta a avaliação do TL entre idosos e adultos com perfis distintos.

A seleção de atributos foi conduzida em duas fases: análise da relevância clínica por especialista médico e análise de importância baseada em resultados obtidos pelo algoritmo Boruta \cite{kursa:2010}.  O \textit{dataset} utilizado contém 50 variáveis clínicas e demográficas de pacientes diagnosticados com DRC (estágios 2 a 5). Oito características preditivas foram selecionadas: idade, pressão sistólica, etiologia da DRC, hemoglobina, albumina, taxa de filtração glomerular estimada, proteinúria, relação de proteína/creatinina urinária, além do atributo alvo. 

Para apoiar a execução de atividades de TL, o domínio de fonte e o domínio de alvo devem compartilhar características relevantes, mesmo que não sejam idênticos. Para analisar tal compartilhamento, as distribuições de atributos clínicos dos subconjuntos \textit{Fonte} e \textit{Alvo} foram calculadas pelo emprego da distância de Jensen-Shannon (JSD) (Figura~\ref{fig:jsd}). Valores próximos de 0 indicam maior similaridade entre as distribuições, enquanto valores mais próximos de 1 representam maior divergência. A análise revelou uma distância global de JSD = 0,2186, sugerindo que as distribuições de variáveis dos conjuntos apresentam similaridade moderada. As variáveis com maior similaridade foram progressão da DRC (JSD = 0,0739) e proteinúria (JSD = 0,0845), enquanto idade (JSD = 0,4271) e pressão sistólica (JSD = 0,3536) apresentaram as maiores divergências, o que era esperado dado o critério de segmentação etária adotado.

\begin{figure}[H]
\centering
\includegraphics[width=0.65\linewidth]{jsd_comparison.png} 
\caption{Distância de Jensen-Shannon entre os \textit{datasets} Fonte e Alvo.}
\label{fig:jsd}
\end{figure}



Foi implementada uma estratégia de TL supervisionado para classificação binária da progressão de DRC entre os \textit{domínios fonte} e \textit{alvo}. O \textit{Modelo Fonte} consistiu em: modelo MLP (16--8--1 neurônios,  ativações \textit{ReLU} nas camadas ocultas e \textit{Sigmóide} na saída), treinado no \textit{domínio fonte} por 20 épocas, com variáveis padronizadas via \textit{StandardScaler} (ajustado no fonte e aplicado ao alvo), partição de 70/30 (treino/validação), otimizador \textit{Adam} (taxa de aprendizado -- LR = 0,01), função de perda \textit{binary cross-entropy} e acurácia como métrica primária.

O desempenho inicial foi avaliado no \textit{domínio alvo} antes do \textit{fine-tuning} via TL, utilizando limiar de 0,5 e as métricas acurácia, precisão, \textit{recall} e \textit{F1-score}. Para adaptação, as duas camadas ocultas foram congeladas e realizou-se \textit{fine-tuning} da camada de saída com dados do alvo (80/20 treino/validação) durante 10 épocas, mantendo os mesmos parâmetros de otimização. Por fim, as métricas foram comparadas no modelo alvo pré e pós \textit{fine-tuning} e sinais de \textit{overfitting} foram analisados pela discrepância treino-validação, a fim de quantificar o ganho de generalização do modelo. 

%A avaliação da eficácia da abordagem investigada ocorreu em três cenários experimentais: (1) \textit{Modelo Fonte}: MLP, treinado e testado exclusivamente com dados do \textit{domínio fonte}, foi testado no \textit{domínio alvo} para avaliar a performance com \textit{dataset} com maior volume de dados, mas de um domínio distinto; (2) \textit{Modelo Alvo}: modelo com a mesma arquitetura do MLP, treinado e testado exclusivamente com os dados do \textit{domínio alvo} para avaliar a performance do treinamento em um domínio similar, mas com \textit{dataset} com menor volume de dados; (3) \textit{Modelo Proposto (TL)}: o modelo MLP foi treinado com os dados do \textit{domínio fonte}; em seguida, uma segunda etapa de treinamento foi realizada visando o ajuste fino do modelo com dados do \textit{domínio alvo}.

A eficácia da abordagem proposta foi avaliada em três cenários experimentais. (1) \textit{Modelo Fonte}: um MLP treinado e validado exclusivamente com dados do \textit{domínio fonte} foi posteriormente testado no \textit{domínio alvo}, permitindo verificar o desempenho em um conjunto de maior volume, porém proveniente de um domínio distinto. (2) \textit{Modelo Alvo}: modelo com a mesma arquitetura do MLP, treinado e validado apenas com dados do \textit{domínio alvo}, possibilitando avaliar o desempenho quando o treinamento ocorre diretamente no domínio de interesse, embora com menor volume de dados. (3) \textit{Modelo Proposto (TL)}: o MLP foi inicialmente treinado com os dados do \textit{domínio fonte} e, em seguida, submetido a uma etapa adicional de fine-tuning utilizando os dados do \textit{domínio alvo}, visando adaptar os parâmetros às suas particularidades.

\section{Resultados} \label{sec:resultados}

O desempenho preditivo dos modelos é apresentado na Tabela \ref{tab:1}.  O \textit{Modelo Fonte} (pré-treinado exclusivamente com dados de idosos) apresentou alta precisão, mas baixo \textit{recall}, evidenciando limitação de generalização ao não identificar mais de 40\% dos casos de progressão de DRC em adultos. O \textit{Modelo Alvo} (treinado apenas com dados de adultos) obteve o \textit{recall} superior, porém à custa de menor precisão, refletindo maior incidência de falsos positivos. O \textit{Modelo Proposto} (baseado em TL) atingiu os maiores valores de acurácia e precisão entre os cenários avaliados, demonstrando melhor equilíbrio entre desempenho discriminativo e capacidade de generalização. 

\begin{table}[H]
\centering
\small

\caption{Métricas de desempenho dos modelos MLP na predição da progressão da DRC.}\label{tab:1}
\begin{tabular}{ p{2.2cm}|p{2.2cm}|p{2.2cm}|p{2.2cm}|p{2.2cm}  }

 \hline
 \textbf{Modelo}&\textbf{Acurácia(\%)}&\textit{\textbf{Precisão(\%)}}&\textbf{Recall(\%)}&\textit{\textbf{F1-Score(\%)}}\\
 \hline
    Fonte           & 84,62 & 90,91 & 58,82 & 71,43 \\
 \hline
    Alvo            & 86,54 & 77,78 & 82,35 & 80,00 \\
 \hline
    Proposto (TL)   & 88,46 & 92,31 & 70,59 & 80,00 \\
 \hline

\end{tabular}
\end{table}

\section{Conclusão} \label{sec:conclusao}

Este trabalho investiga a aplicação de TL em um modelo MLP construído a partir de dados tabulares na predição de progressão da DRC em populações com diferentes perfis etários. Embora o \textit{Modelo Alvo} tenha alcançado um \textit{recall} superior, o \textit{Modelo Proposto} (TL) obteve a maior acurácia e uma precisão notavelmente elevada. Estes resultados sugerem que o pré-treinamento no \textit{domínio fonte} permitiu a extração de características fundamentais da DRC, enquanto a etapa de ajuste fino adaptou o modelo às particularidades do \textit{domínio alvo}. Isso reforça a relevância da segmentação etária na modelagem preditiva e evidencia o potencial de TL no apoio à construção de soluções de IA aplicadas à tomada de decisão clínica, especialmente em cenários de dados heterogêneos e limitados. 


\bibliographystyle{sbc}
\bibliography{sbc-template}

\end{document}
